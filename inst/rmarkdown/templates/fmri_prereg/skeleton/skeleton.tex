\documentclass[]{article}
    \usepackage{lmodern}
    \usepackage{amssymb,amsmath}
\usepackage{ifxetex,ifluatex}
\usepackage{fixltx2e} % provides \textsubscript
\ifnum 0\ifxetex 1\fi\ifluatex 1\fi=0 % if pdftex
\usepackage[T1]{fontenc}
\usepackage[utf8]{inputenc}
  \else % if luatex or xelatex
\ifxetex
\usepackage{mathspec}
\usepackage{xltxtra,xunicode}
\else
  \usepackage{fontspec}
\fi
\defaultfontfeatures{Mapping=tex-text,Scale=MatchLowercase}
\newcommand{\euro}{€}
        \fi
% use upquote if available, for straight quotes in verbatim environments
\IfFileExists{upquote.sty}{\usepackage{upquote}}{}
% use microtype if available
\IfFileExists{microtype.sty}{%
  \usepackage{microtype}
  \UseMicrotypeSet[protrusion]{basicmath} % disable protrusion for tt fonts
}{}
  \usepackage[right=1in,top=1.25in,bottom=1.25in,left=2.5in]{geometry}
  \ifxetex
\usepackage[setpagesize=false, % page size defined by xetex
            unicode=false, % unicode breaks when used with xetex
            xetex]{hyperref}
\else
  \usepackage[unicode=true]{hyperref}
\fi
\hypersetup{breaklinks=true,
bookmarks=true,
pdfauthor={},
pdftitle={My fMRI preregistration},
colorlinks=true,
citecolor=blue,
urlcolor=blue,
linkcolor=magenta,
pdfborder={0 0 0}}
\urlstyle{same}  % don't use monospace font for urls
\setlength{\parindent}{0pt}
\setlength{\parskip}{6pt plus 2pt minus 1pt}
\setlength{\emergencystretch}{3em}  % prevent overfull lines
\providecommand{\tightlist}{%
\setlength{\itemsep}{0pt}\setlength{\parskip}{0pt}}
\setcounter{secnumdepth}{0}

% Customization for cos_prereg
\usepackage{longtable,booktabs,threeparttable,tabularx}
\linespread{1.5}
\newcounter{question}
\setcounter{question}{0}

%%% Use protect on footnotes to avoid problems with footnotes in titles
\let\rmarkdownfootnote\footnote%
\def\footnote{\protect\rmarkdownfootnote}

%%% Change title format to be more compact
\usepackage{titling}

\def\changemargin#1#2{\list{}{\rightmargin#2\leftmargin#1}\item[]}
\let\endchangemargin=\endlist

% Create subtitle command for use in maketitle
\newcommand{\subtitle}[1]{
\posttitle{
\begin{center}\large#1\end{center}
}
}

\setlength{\droptitle}{-2em}
\title{My fMRI preregistration}
\pretitle{\begin{changemargin}{-8pc}{0pc} \centering\large Preregistration\\ \Huge}
\posttitle{\end{changemargin}}
  \author{
          First Author\textsuperscript{1},
          Ernst-August Doelle\textsuperscript{1,2}          \\ \vspace{0.5cm}
              \textsuperscript{1} Wilhelm-Wundt-University\\
              \textsuperscript{2} Konstanz Business School      }

  \def\affdep{{"", ""}}%
  \def\affcity{{"", ""}}%
  \preauthor{\begin{changemargin}{-8pc}{0pc} \centering\large}
  \postauthor{\end{changemargin}}
\date{17. September 2019}
\predate{\begin{changemargin}{-8pc}{0pc} \centering\large\emph}
\postdate{\end{changemargin}}
\usepackage{fancyhdr}
\pagestyle{fancy}
\renewcommand{\headrulewidth}{0pt}
\lhead{}
\rhead{\large\textsc{\MakeLowercase{My preregistration}}}



% Title settings
\usepackage{titlesec}
\titleformat{\section}[display]{\bfseries\Large}{\thesection}{}{}[]
\titlespacing{\section}{0pc}{*3}{*1.5}
\titleformat{\subsection}[leftmargin]{\titlerule\bfseries\filleft}{\thesubsection}{.5em}{}
\titlespacing{\subsection}{8pc}{5ex plus .1ex minus .2ex}{1.5pc}
  

% Redefines (sub)paragraphs to behave more like sections
\ifx\paragraph\undefined\else
\let\oldparagraph\paragraph
\renewcommand{\paragraph}[1]{\oldparagraph{#1}\mbox{}}
\fi
\ifx\subparagraph\undefined\else
\let\oldsubparagraph\subparagraph
\renewcommand{\subparagraph}[1]{\oldsubparagraph{#1}\mbox{}}
\fi


\begin{document}
\maketitle
\vspace{2pc}


\newcommand\Question[2]{%
   \leavevmode\par
   \stepcounter{question}
   \noindent
   \textbf{\thequestion. #1}. #2\par}

\newcommand\Answer[1]{%
    \noindent
    \textit{Registered response}: #1\par}
    
\hypertarget{project-title}{%
\subsection{Project Title}\label{project-title}}

Enter your response here.

\hypertarget{introduction}{%
\subsection{Introduction}\label{introduction}}

Enter your response here.

\hypertarget{aims-hypotheses}{%
\subsection{Aims \& Hypotheses}\label{aims-hypotheses}}

Enter your response here.

\hypertarget{existing-data}{%
\subsection{Existing Data}\label{existing-data}}

\textbf{Registration prior to creation of data}

\textbf{Registration prior to any human observation of the data}

\textbf{Registration prior to accessing the data}

\textbf{Registration prior to analysis of the data}

\textbf{Registration following analysis of the data}

\hypertarget{explanation-of-existing-data}{%
\subsection{Explanation of Existing
Data}\label{explanation-of-existing-data}}

Enter your response here.

\hypertarget{details-of-larger-study}{%
\subsection{Details of Larger Study}\label{details-of-larger-study}}

Is your preregistration part of a larger project?

\textbf{Yes}

\textbf{No}

\hypertarget{data-collection-procedures}{%
\subsection{Data Collection
Procedures}\label{data-collection-procedures}}

\textbf{Population}

Enter your response here.

\textbf{Recruitment efforts}

Enter your response here.

\textbf{Inclusion/Exclusion criteria}

Enter your response here.

\textbf{Clinical criteria (if applicable)}

Enter your response here.

\textbf{Matching strategy (if applicable)}

Enter your response here.

\textbf{Payment for participation}

Enter your response here.

\textbf{IRB, consent/assent obtained}

Enter your response here.

\textbf{Number of subjects participated and analyzed}

Enter your response here.

\textbf{Age}

Enter your response here.

\textbf{Sex}

Enter your response here.

\textbf{Handedness}

Enter your response here.

\textbf{For group comparisons, what variables (if any) were equated
across groups?}

Enter your response here.

\textbf{Study timeline (e.g., number of visits, length of visits, what
was measured/collected at each visit)}

Enter your response here.

\hypertarget{sample-size-stopping-rule}{%
\subsection{Sample Size \& Stopping
Rule}\label{sample-size-stopping-rule}}

\textbf{Target sample size:}

To obtain our target sample size, we plan to recruit: Enter your
response here.

\textbf{Justification of sample size:}

Power analyses (e.g., Neuropowertools, fmri power)

From Nichols et al., 2016, include:

\begin{itemize}
\tightlist
\item
  Effect size: Enter your response here.
\item
  Source of predicted effect size (prior lit, pilot etc.): Enter your
  response here.
\item
  Significant level: Enter your response here.
\item
  Target power: Enter your response here.
\item
  Specify the type of outcome used as the basis of power computations,
  e.g.~signal in a prespecified ROI, or whole image voxelwise (or
  clusterwise, peakwise, etc.): Enter your response here.
\end{itemize}

\textbf{Stopping rule:}

\begin{itemize}
\tightlist
\item
  Time constraints : Enter your response here.
\item
  Money constraints : Enter your response here.
\item
  Personnel constraints : Enter your response here.
\end{itemize}

\textbf{Contingencies for if your target sample size is not met :} Enter
your response here.

\hypertarget{measured-behavioral-variables}{%
\subsection{Measured Behavioral
Variables}\label{measured-behavioral-variables}}

\textbf{Outcome measures}

Enter your response here.

\textbf{Predictor measures}

Enter your response here.

\textbf{Covariate measures}

Enter your response here.

\textbf{How was behavioral task performance measured}

Enter your response here.

\textbf{Contingency plans for behavioral analysis}

Enter your response here.

\begin{quote}
E.g., If the X questionnaire is missing for more than 10\% of
participants we will not use it or if X does not show variability in
response (either ceiling or floor effects) in which we cannot look at
behavioral pattern of interest, we will not use that questionnaire and
use Y questionnaire instea
\end{quote}

\hypertarget{additional-operational-definitions}{%
\subsection{Additional Operational
Definitions}\label{additional-operational-definitions}}

\textbf{Region Specificity}:

Enter your response here.

\textbf{Any other definitions used across study}:

Enter your response here.

\hypertarget{transformations}{%
\subsection{Transformations}\label{transformations}}

Enter your response here.

\textbf{Contingency plans for transformation}:

Enter your reponse here.

\textbf{Code, if applicable}:

Enter your response here.

\hypertarget{analysis-data-exclusion}{%
\subsection{Analysis Data Exclusion}\label{analysis-data-exclusion}}

\textbf{Outliers}

Enter your response here.

\textbf{Reasons for possible rejection}

\textbf{Contingency plans}

\textbf{Dealing with incomplete/missing data}

\hypertarget{experimental-design}{%
\subsection{Experimental Design}\label{experimental-design}}

\textbf{Design Specifications}

\begin{itemize}
\item
  Design type: Enter your response here.
\item
  Conditions \& Stimuli: Enter your response here.
\item
  Number of blocks, trials or experimental units per session and/or
  subject: Enter your response here.
\item
  Timing and Duration: Enter your response here.
\item
  Length of experiment: Enter your response here.
\item
  Was the design optimized for efficiency, and if so, how? Enter your
  response here.
\item
  Presentation software: Enter your response here.
\end{itemize}

\textbf{Task Specification}

\begin{itemize}
\item
  Instructions to subjects: Enter your response here.
\item
  Stimuli: Enter your response here.
\item
  Stimuli presentation \& response collection
  Randomization/pseudo-randomized: Enter your response here.
\item
  Run order: Enter your response here.
\end{itemize}

\hypertarget{data-acquisition}{%
\subsection{Data acquisition}\label{data-acquisition}}

\textbf{Subject Preparation}

\begin{itemize}
\item
  Mock scanning: Enter your response here.
\item
  Specific accommodations: Enter your response here.
\item
  Experimental personnel: Enter your response here.
\end{itemize}

\textbf{MRI system}

\begin{itemize}
\tightlist
\item
  Manufacturer, field strength , model name: Enter your response here.
\end{itemize}

\textbf{MRI acquisition}

\begin{itemize}
\item
  Pulse sequence: Enter your response here.
\item
  Image type: Enter your response here.
\end{itemize}

\hypertarget{essential-sequence-imaging-parameters-for-all-acquisitions--echo-time-te.--repetition-time-tr.o-for-multishot-acquisitions-additionally-the-time-per-volume.--flip-angle-fa.--acquisition-time-duration-of-acquisition.-functional-mri--number-of-volumes.--sparse-sampling-delay-delay-in-tr-if-used.-inversion-recovery-sequences--inversion-time-ti.-b0-field-maps--echo-time-difference-dte.-diffusion-mri--number-of-directions.o-direction-optimization-if-used-and-type.--b-values.--number-of-b0-images.--number-of-averages-if-any.--single-shell-multishell-specify-equal-or-unequal-spacing.--single-or-dualspinecho-gradient-mode-serial-or-parallel.--if-cardiac-gating-used.imaging-parameters--field-of-view.--inplane-matrix-size-slice-thickness-and-interslice-gap-for-2d-acquisitions.---slice-orientation--axial-sagittal-coronal-or-oblique.--angulation-if-acquistion-not-aligned-with-scanner-axes-specifyangulation-to-acpc-line-see-slice-position-procedure.---3d-matrix-size-for-3d-acquisitions.phase-encodingparallel-imaging-method-parametersmultiband-parametersreadout-parameters-fat-suppression-for-anatomical-state-if-usedshimmingslice-order-timingbrain-coverage-e.g.-whole-brain-was-cerebellum-brain-stem-includedscanner-side-preprocessing-e.g.-including-reconstruction-matrix-size-differing-from-acquisition-matrix-size-prospective-motion-correction-including-details-of-any-optical-tracking-and-how-motion-parameters-are-used-signal-inhomogeneity-correction-distortion-correction.}{%
\subsubsection{\texorpdfstring{Essential sequence \& imaging parameters
For all acquisitions:- Echo time (TE).- Repetition time (TR).o For
multishot acquisitions, additionally the time per volume.- Flip angle
(FA).- Acquisition time (duration of acquisition). Functional MRI:-
Number of volumes.- Sparse sampling delay (delay in TR) if used.
Inversion recovery sequences:- Inversion time (TI). B0 field
\url{maps:-} Echo time difference (dTE). Diffusion MRI:- Number of
directions.o Direction optimization, if used and type.- b-values.-
Number of b=0 images.- Number of averages (if any).- Single shell,
multishell (specify equal or unequal spacing).- Single or dualspinecho,
gradient mode (serial or parallel).- If cardiac gating used.Imaging
parameters:- Field of view.- Inplane matrix size, slice thickness and
interslice gap, for 2D acquisitions. - Slice orientation:- Axial,
sagittal, coronal or oblique.- Angulation: If acquistion not aligned
with scanner axes, specifyangulation to ACPC line (see Slice position
procedure). - 3D matrix size, for 3D acquisitions.Phase encodingParallel
imaging method \& parametersMultiband parametersReadout parameters Fat
suppression (for anatomical, state if used)ShimmingSlice order \&
timingBrain coverage (e.g., whole brain, was cerebellum, brain stem
included)Scanner-side preprocessing (e.g., Including: Reconstruction
matrix size differing from acquisition matrix size; Prospective-motion
correction (including details of any optical tracking, and how motion
parameters are used); Signal inhomogeneity correction;
Distortion-correction.)}{Essential sequence \& imaging parameters For all acquisitions:- Echo time (TE).- Repetition time (TR).o For multishot acquisitions, additionally the time per volume.- Flip angle (FA).- Acquisition time (duration of acquisition). Functional MRI:- Number of volumes.- Sparse sampling delay (delay in TR) if used. Inversion recovery sequences:- Inversion time (TI). B0 field maps:- Echo time difference (dTE). Diffusion MRI:- Number of directions.o Direction optimization, if used and type.- b-values.- Number of b=0 images.- Number of averages (if any).- Single shell, multishell (specify equal or unequal spacing).- Single or dualspinecho, gradient mode (serial or parallel).- If cardiac gating used.Imaging parameters:- Field of view.- Inplane matrix size, slice thickness and interslice gap, for 2D acquisitions. - Slice orientation:- Axial, sagittal, coronal or oblique.- Angulation: If acquistion not aligned with scanner axes, specifyangulation to ACPC line (see Slice position procedure). - 3D matrix size, for 3D acquisitions.Phase encodingParallel imaging method \& parametersMultiband parametersReadout parameters Fat suppression (for anatomical, state if used)ShimmingSlice order \& timingBrain coverage (e.g., whole brain, was cerebellum, brain stem included)Scanner-side preprocessing (e.g., Including: Reconstruction matrix size differing from acquisition matrix size; Prospective-motion correction (including details of any optical tracking, and how motion parameters are used); Signal inhomogeneity correction; Distortion-correction.)}}\label{essential-sequence-imaging-parameters-for-all-acquisitions--echo-time-te.--repetition-time-tr.o-for-multishot-acquisitions-additionally-the-time-per-volume.--flip-angle-fa.--acquisition-time-duration-of-acquisition.-functional-mri--number-of-volumes.--sparse-sampling-delay-delay-in-tr-if-used.-inversion-recovery-sequences--inversion-time-ti.-b0-field-maps--echo-time-difference-dte.-diffusion-mri--number-of-directions.o-direction-optimization-if-used-and-type.--b-values.--number-of-b0-images.--number-of-averages-if-any.--single-shell-multishell-specify-equal-or-unequal-spacing.--single-or-dualspinecho-gradient-mode-serial-or-parallel.--if-cardiac-gating-used.imaging-parameters--field-of-view.--inplane-matrix-size-slice-thickness-and-interslice-gap-for-2d-acquisitions.---slice-orientation--axial-sagittal-coronal-or-oblique.--angulation-if-acquistion-not-aligned-with-scanner-axes-specifyangulation-to-acpc-line-see-slice-position-procedure.---3d-matrix-size-for-3d-acquisitions.phase-encodingparallel-imaging-method-parametersmultiband-parametersreadout-parameters-fat-suppression-for-anatomical-state-if-usedshimmingslice-order-timingbrain-coverage-e.g.-whole-brain-was-cerebellum-brain-stem-includedscanner-side-preprocessing-e.g.-including-reconstruction-matrix-size-differing-from-acquisition-matrix-size-prospective-motion-correction-including-details-of-any-optical-tracking-and-how-motion-parameters-are-used-signal-inhomogeneity-correction-distortion-correction.}}

Scan duration (in seconds)Other non-standard proceduresT1 stabilization
(discarded ``dummy'' scans acquired discarded by scanner)Diffusion MRI
gradient table (Also referred to as the bmatrix, but not to be confused
with the 3×3 matrix that describes diffusion weighting for a single
diffusion weighted measurement)Perfusion: Arterial Spin Labelling MRI
ASL Labelling method (e.g.~continuous ASL (CASL), pseudocontinuous ASL
(PCASL), Pulsed ALS (PASL), velocity selective ASL (VSASL).- Use of
background suppression pulses and their timing.- For either PCASL or
CASL report: Label Duration.- Postlabeling delay (PLD).- Location of the
labeling plane.- For PCASL also report:- Average labeling gradient.-
Sliceselective labeling gradient.- Flip angle of B1 pulses.- Assessment
of inversion efficiency; QC used to ensureoffresonance artifacts not
problematic, signal obtained over wholebrain.- For CASL also report:
oUse of a separate labeling coil.- Control scan/pulse used.- B1
amplitude.- For PASL report - TI.- Labeling slab thickness.- Use of
QUIPSS pulses and their timing.- For VSASL- TI.- Choice of velocity
selection cutoff (``VENC'').Perfusion: Dynamic Susceptibility Contrast
MRI Specify:- Number of baseline volumes.- Type, name and manufacturer
of intravenous bolus (e.g.~gadobutrol,Gadavist, Bayer).- Bolus amount
and concentration (e.g.~0.1 ml/kg and 0.1 mmol/kg). - Injection rate
(e.g.~5 ml/s).- Postinjection of saline (e.g.~20 ml).- Injection method
(e.g.~power injector).Recommendations of fMRI details come from Nichols
et al., 2016; Poldrack et al., 2008.For extended checklist guidelines
for this section following data analysis, see PHBM COBIDAS report
Nichols et al., 2016. PreprocessingCan fill in the table or write
paragraph below as you would for paper and use table as checklist of
topics covered. fMRI preregistration template \textbar{} Jessica
FlanneryPreliminary quality controlMotion monitoring (For functional or
diffusion acquisitions, any visual or quantitative checks for severe
motion; likewise, for structural images, checks on motion or general
image quality.)Incidental findings (Protocol for review of any
incidental findings, and how they are handled in particular with respect
to possible exclusion of a subject's data.)Data preprocessingFor each
piece of software used, give the version number (or, if no version
number is available, date of last application of updates)If any subjects
required different processing operations or settings in the analysis,
those differences should be specified explicitlyPre-processing:
generalSpecify order of preprocessing operationsDescribe any data
quality control measuresUnwarping of B0 distortionsSlice timing
correctionReference slice and type of interpolation used (e.g., ``Slice
timing correction to the first slice as performed, using SPM5's Fourier
phase shift interpolation'')Motion correctionReference scan, image
similarity metric, type of interpolation used, degrees-of-freedom (if
not rigid body) and, ideally, optimization method, e.g., ``Head motion
corrected with FSL's MCFLIRT by maximizing the correlation ratio between
each timepoint and the middle volume, using linear
interpolation.''Motion susceptibility correction usedSmoothingSize and
type of smoothing kernel (provide justification for size; e.g., for a
group study,``12 mm FHWM Gaussian smoothing applied to ameliorate
differences in intersubject localization''; for single subject fMRI ``6
mm FWHM Gaussian smoothing used to reduce noise'')Intersubject
registrationIntersubject registration method usedIllustration of the
voxels present in all subjects (``mask image'') can be helpful,
particularly for restricted fields of view (to illustrate overlap of
slices across all subjects). Better still would be an indication of
average BOLD sensitivity within each voxel in the maskTransformation
model and optimizationTransformation model (linear/affine, nonlinear),
type of any non-linear transformations (polynomial, discrete cosine
basis), number of parameters (e.g., 12 parameter affine, 3 × 2 × 3 DCT
basis), regularization, image-similarity metric, and interpolation
methodObject image information (image used to determine transformation
to atlas)Anatomical MRI? Image properties (see above)Co-planar with
functional acquisition?Functional acquisition co-registered to
anatomical? if so, how?Segmented gray image? fMRI preregistration
template \textbar{} Jessica FlanneryFunctional image (single or
mean)Atlas/target informationBrain image template space, name, modality
and resolution (e.g., ``FSL's MNI Avg152, T1 2 × 2 × 2 mm''; ``SPM2's
MNI gray matter template 2 × 2 × 2 mm'')Coordinate space (Typically MNI,
Talairach, or MNI converted to TalairachIf MNI converted to Talairach,
what method? e.g., Brett's mni2tal?How were anatomical locations (e.g.,
gyral anatomy, Brodmann areas) determined? (e.g., paper atlas, Talairach
Daemon, manual inspection of individuals' anatomy, etc.)SmoothingSize
and type of smoothing kernel (provide justification for size; e.g., for
a group study, ``12 mm FHWM Gaussian smoothing applied to ameliorate
differences in intersubject localization''; for single subject fMRI ``6
mm FWHM Gaussian smoothing used to reduce noise'')Recommendations of
fMRI details come from Nichols et al., 2016; Poldrack et al., 2008.For
extended checklist guidelines for this section following data analysis,
see PHBM COBIDAS report Nichols et al., 2016. Statistical modelingFor
all prompts and tables, can fill in the table or write paragraph below
as you would for paper and use table as checklist of topics
covered.Planned comparison If the experiment has multiple conditions,
what are the specific planned comparisons, or is an omnibus ANOVA
used?General issuesFor novel methods that are not described in detail in
a separate paper, provide explicit descriptionand validation of
method.Remember to include package and package version used for each
test.First level (fx) modelingEventrelated design predictors.- Modeled
duration, if other than zero.- Parametric modulation.Block Design
predictors.(Note whether baseline was explicitly modeled.)HRF basis,
typically one of:Canonical only.Canonical plus temporal
derivative.Canonical plus temporal and dispersion derivative. Smooth
basis (e.g.~SPM ``informed'' or Fourier basis; FSL's FLOBS).Finite
Impulse Response model.Drift regressors (e.g.~DCT basis in SPM, with
specified cutoff). fMRI preregistration template \textbar{} Jessica
FlanneryMovement regressors; specify if squares and/or temporal
derivative used.Any other nuisance regressors, and whether they were
entered as interactions (e.g.~with a task effect in 1st level fMRI, or
with group effect).Any orthogonalization of regressors, and set of other
regressors used to orthogonalize against.Contrast construction (Exactly
what terms are subtracted from what? Define these in terms of task or
stimulus conditions (e.g., using abstract names such as AUDSTIM,
VISSTIM) instead ofunderlying psychological concepts.Autocorrelation
model type (e.g., AR(1), AR(1) + WN, or arbitrary autocorrelation
function), and whether global or local.(e.g., for SPM2/SPM5,
`Approximate AR(1) autocorrelation model estimated at omnibus
F-significant voxels (P \textless{} 0.001), used globally over the whole
brain'; for FSL, `Autocorrelation function estimated locally at each
voxel, tapered and regularized in space.').Second level (group)
modelingStatistical model and estimation method, inference type
(mixed/random effects or fixed), e.g., ``Mixed effects inference with
one sample t-test on summary statistic'' (SPM2/SPM5), e.g., ``Mixed
effects inference with Bayesian 2-level model with fast approximation to
posterior probability of activation.'' (FSL)If fixed effects inference
used, justifyIf more than 2-levels, describe the levels and assumptions
of the model (e.g., are variances assumed equal between groups)Repeated
measures? If multiple measurements per subject, list method to account
for within subject correlation, exact assumptions made about
correlation/variancee.g., SPM: ``Within-subject correlation estimated at
F-significant voxels (P \textless{}0.001), then used globally over whole
brain''; or, if variances for each measure are allowed to vary,
``Within-subject correlation and relative variance
estimated\ldots{}''For group model with repeated measures, specify:- How
condition effects are modeled (e.g.~as factors, or as linear trends).-
Whether subject effects are modeled (i.e.~as regressors, as opposed
towith a covariance structure).For group effects: clearly state whether
or not covariates are split by group (i.e.~fit as a groupby-covariate
interaction).Model type (Some suggested terms include:- ``Mass
Univariate''.- ``Multivariate'' (e.g.~ICA on whole brain data).- ``Mass
Multivariate'' (e.g.~MANOVA on diffusion or morphometry tensordata).-
``Local Multivariate'' (e.g.~``searchlight'').- ``Multivariate,
intrasubject predictive'' (e.g.~classify individual trials
ineventrelated fMRI).- ``Multivariate intersubject predictive''
(e.g.~classify subjects as patient vs.control).- ``Representational
Similarity Analysis''.)Model settings (The essential details of the
model. For massunivariate, first level fMRI, these include:- Drift
model, if not already specified as a dependent variable (e.g.~locally
linear fMRI preregistration template \textbar{} Jessica
Flannerydetrending of data \& regressors, as in FSL).- Autocorrelation
model (e.g.~global approximate AR(1) in SPM; locally regularized
autocorrelation function in FSL).For massunivariate second level fMRI
these include:- Fixed effects (all subjects' data in one model).- Random
or mixedeffects model, implemented with:- Ordinary least squares (OLS,
aka unweighted summary statistics approach; SPM default, FSL FEAT's
``Simple OLS'').- weighted least squares (i.e.~FSL FEAT's ``FLAME 1''),
using voxelwise estimate of between subject variance.- Global weighted
least squares (i.e.~SPM's MFX).With any group (multisubject) model,
indicate any specific variance structure, e.g.- Unequal variance between
groups (and if globally pooled, as in SPM).- If repeated measures, the
specific covariance structure assumed (e.g.compound symmetric, or
arbitrary; if globally pooled).For localmultivariate report:- The number
of voxels in the local model.- Local model used (e.g.~Canonical
Correlation Analysis) with anyconstraints (e.g.~positive weights
only).ROI analysisHow were ROIs defined (e.g., functional, anatomical,
parcel localizer)?How was signal extracted within ROI?(e.g., average
parameter estimates, FIR deconvolution?)If percent signal change
reported, how was scaling factor determined (e.g., height of block
regressor or height of isolated event regressor)?Is change relative to
voxel-mean, or whole-brain mean?Justify definition of ROI and analysis
conducted with it: (e.g., if your ROI is defined based on the cluster;
how will you ensure your ROI analyses are not circular?)If not
previously specified above, what statistical model will you use to test
each hypothesis? Please include the type of model (e.g.~ANOVA, multiple
regression, SEM, etc) and the specification of the model (this includes
each variable that will be included as predictors, outcomes, or
covariates). Please specify any interactions that will be tested and
remember that any test not included here must be noted as an exploratory
test in your final article.Recommendations of fMRI details come from
Nichols et al., 2016; Poldrack et al., 2008.For extended checklist
guidelines for this section following data analysis, see PHBM COBIDAS
report Nichols et al., 2016.

\hypertarget{references}{%
\section{References}\label{references}}

\hypertarget{section}{%
\subsection{}\label{section}}

\vspace{-2pc}
\setlength{\parindent}{-0.5in}
\setlength{\leftskip}{-1in}
\setlength{\parskip}{8pt}

\noindent

\end{document}